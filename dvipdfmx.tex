\documentclass[a4paper,xetex,oneside]{book}
\usepackage{xltxtra}
\usepackage{fontspec}
\usepackage{microtype}
\usepackage{xunicode}
\usepackage{unicode-math}
%%%%%%%%%%%%%%%%%%%%%%%%%%%%%%%%%%%%%%%%%%%%%%%%%%%%%%%%%%%%%%%%%%%%%%%%%%
%% PLATFORM DEPENDENT INSTRUCTION HERE
%% Font setup for body text.
\setmainfont{Constantia}
\setsansfont{Cambria}
\setmonofont[Scale=0.92]{Consolas}% Adjust xheight difference
\setmathfont{Cambria Math}
%% The following section is for showing some fancy examples.
%% Some fonts used here may not be availabel on your system.
%% Please modify this. Just replacing with empty macros is OK.
\newcommand{\jpzerofourexamples}{{%
\setmainfont[Scale=5,RawFeature=+jp04]{SourceHanSerifJP-Light.otf}葛祇逢}}
\newcommand{\jpninezeroexamples}{{%
\setmainfont[Scale=5,RawFeature=+jp90]{SourceHanSerifJP-Light.otf}葛祇逢}}
%% We use PostScript raw code here to test dvipdfmx's capability.
%% TFM files `uprml' and `uprmlv' are distributed with upTeX.
%% TrueType font `yumindb.ttf' is bundled with Windows 10.
\newcommand{\jphoritext}{%
\makebox[112bp][l]{%
\raisebox{88bp}[112bp][0bp]{%
\special{pdf:mapline uprml UniJIS-UTF8-H yumindb.ttf}
\special{ps: uprml findfont 16 scalefont setfont
  currentpoint moveto (「こんにちは」) show}%
}}}
\newcommand{\jpverttext}{%
\makebox[16bp][l]{%
\raisebox{112bp}[112bp][0bp]{%
\special{pdf:mapline uprmlv UniJIS-UTF8-V yumindb.ttf}
\special{ps: uprmlv findfont 16 scalefont setfont
  currentpoint moveto (「こんにちは」) show}%
}}}
%%%%%%%%%%%%%%%%%%%%%%%%%%%%%%%%%%%%%%%%%%%%%%%%%%%%%%%%%%%%%%%%%%%%%%%%%
\usepackage{xcolor,hyperref,hyperxmp}
\hypersetup{%
    bookmarksnumbered=true,%
    colorlinks=true,%
    urlcolor=[rgb]{0.2,0.2,0.4},
    linkcolor=[rgb]{0.2,0.2,0.4},
    citecolor=[rgb]{0.2,0.2,0.4},
    pdftitle={The Dvipdfmx User's Manual},%
    pdfauthor={The dvipdfmx project team},%
    pdfsubject={A User's Manual for Dvipdfmx and Xdvipdfmx.},%
    pdfkeywords={dvipdfmx, XeTeX, TeX, LaTeX},
    pdflang=en,
    pdfcopyright={Copyright © The dvipdfmx project team.
    Permission is granted to copy, distribute and/or modify this document
    under the terms of the GNU Free Documentation License, Version 1.3
    or any later version published by the Free Software Foundation;
    with no Invariant Sections, no Front-Cover Texts, and no Back-Cover Texts.
    A copy of the license is included in the section entitled "GNU
    Free Documentation License".},
    pdflicenseurl={http://www.gnu.org/licenses/fdl.html}
}
\usepackage{listings}
\lstset{
    keepspaces=true,
    basicstyle={\ttfamily},
    frame={tb},
    breaklines=true,
    columns=[l]{fullflexible},
    numbers=none,
    xrightmargin=2em,
    xleftmargin=2em,
    aboveskip=2em,
    belowskip=2em
}
\usepackage{mflogo}
\usepackage{lipsum}
\usepackage{array}
\usepackage{marginnote}
\renewcommand*{\marginfont}{\footnotesize\itshape}
\usepackage[noorphans,font=itshape]{quoting}
\newcommand{\package}[1]{\textit{#1}}
\newcommand{\code}[1]{\mbox{\texttt{#1}}}
\newcommand{\keyword}[1]{\textit{#1}}
\newcommand{\option}[1]{\mbox{`\texttt{#1}'}}
\newcommand{\dvipdfm}{\texttt{dvipdfm}}
\newcommand{\dvipdfmx}{\texttt{dvipdfmx}}
\newcommand{\xdvipdfmx}{\texttt{xdvipdfmx}}
\newcommand{\deprecated}[1]{\marginnote{\addfontfeatures{Color=CC3333}#1}}
\newcommand{\newfeature}[1]{\marginnote{\addfontfeatures{Color=3366CC}#1}}
\newcommand{\lnum}[1]{{\addfontfeatures{RawFeature=+lnum}#1}}
% For placeing drawings via \special
\newcommand{\specialbox}[3]{%
\makebox[#1][l]{\raisebox{-#2}[0bp][#2]{\special{#3}}}}
\usepackage{fancyhdr}
\pagestyle{fancy}
%%DVIPDFMX SETTINGS
\AtBeginDocument{%
  \special{dvipdfmx:config O 1}%
  \special{dvipdfmx:config V 7}
}
\begin{document}

\begin{titlepage}
  \begin{raggedleft}
    {\Huge\bfseries The Dvipdfmx User's Manual}\\[\baselineskip]
    \Large The Dvipdfmx Project Team\\
    August 30, 2018\par
  \end{raggedleft}
\end{titlepage}

\tableofcontents

\chapter{Getting Started}

\section{Introduction}

The \dvipdfmx\ (formerly \dvipdfm-cjk) project provides an extended version of
the \dvipdfm, a DVI to PDF translator developed by Mark~A.~Wicks.

The primary goal of this project is to support multi-byte character encodings
and large character sets such as for East Asian languages.
This project started as a combined work of the \dvipdfm-jpn project by
Shunsaku Hirata and its modified one, \dvipdfm-kor, by Jin-Hwan Cho.

Extensions to \dvipdfm\ include,
\begin{itemize}
    \item Support for OpenType and TrueType font, including partial support
    for the OpenType Layout for finding glyph variants and vertical writing.
    \item Support for CJK-\LaTeX\ and H\LaTeX\ with Subfont Definition Files.
    \item Support for various legacy multi-byte encodings via PostScript CMap
     Resources.
    \item Unicode related features: Unicode as an input encoding and
    auto-creation of ToUnicode CMaps.
    \item Support for p\TeX\ (a Japanese localized variant of \TeX) including
    vertical writing extension.
    \item Some extended DVI specials.
    \item Reduction of output files size with on-the-fly Type1 to CFF (Type1C)
    conversion and PDF object stream.
    \item Advanced raster image support including alpha channels, embedded
    ICC profiles, 16-bit bit-depth colors, and so on.
    \item Basic PDF password security support for PDF output.
\end{itemize}
Some important features are still missing:
\begin{itemize}
    \item Linearization.
    \item Color Management.
    \item Resampling of images.
    \item Selection of compression filters.
    \item Variable font and OpenType 1.8.
    \item and many more...
\end{itemize}

\dvipdfmx\ is now maintained as part of \TeX\ Live. Latest source code can
be found at the \TeX\ Live SVN repository. For an instruction on accessing the
development sources for \TeX\ Live, see,\medskip

\url{http://www.tug.org/texlive/svn/}
\medskip

This document, ``The \dvipdfmx\ User's Manual'', was originally prepared for
\TeX\ Live 2017. Current maintainer of this document is Shunsaku Hirata.
Latest version and contact information can be found at:\medskip

\url{http://github.com/shirat74/dvipdfm-x-doc}
\medskip

\noindent{}Please send questions or suggestions.

\subsection{\xdvipdfmx}

\xdvipdfmx\ is an extended version of \dvipdfmx, and is now incorporated into
\dvipdfmx.

The \xdvipdfmx\ extensions provides support for the Extended DVI (.xdv) format
generated by \XeTeX\ which includes support for platform-native fonts and the
\XeTeX\ graphics primitives, as well as Unicode text and OpenType font.

\XeTeX\ originally used a Mac-specific program called \code{xdv2pdf} as a
backend program instead of \xdvipdfmx. The \code{xdv2pdf} program supported
a couple of special effects that are not yet available through \xdvipdfmx:
The Quartz graphics-based shadow support, AAT ``variation'' fonts such as Skia,
transparency as an attribute of font, and so on.
It would be nice if they continue to be supported. Suggestions and help are
welcomed.


\subsection{Legal Notice}

Copyright © The dvipdfmx project team.
Permission is granted to copy, distribute and/or modify this document
under the terms of the GNU Free Documentation License, Version 1.3
or any later version published by the Free Software Foundation;
with no Invariant Sections, no Front-Cover Texts, and no Back-Cover Texts.
A copy of the license is included in the section entitled
``\hyperref[SEC:FDL]{GNU Free Documentation License}''.

\section{Installation and Usage}

Typical usage and installation steps are not different from the original
\dvipdfm. Please refer documents from \dvipdfm\ distribution for detailed
instruction on how to install and how to use \dvipdfm. The \dvipdfm\ manual is
available from its CTAN site:\medskip

\url{http://www.ctan.org/tex-archive/dviware/dvipdfm}
\medskip

The minimal requirements for building \dvipdfmx\ is the \keyword{kpathsea} library.
the \keyword{zlib} library for compression and the \keyword{libpng} library for PNG
inclusion are highly recommended.
Optionally, the \keyword{libpaper} library might be used to handle paper size.

This document mainly focuses on the additions and modifications to \dvipdfm.
Please refer the
``\href{http://mirrors.ctan.org/dviware/dvipdfm/dvipdfm.pdf}{Dvipdfm User's Manual}''
available from the CTAN site mentioned above for basic usage.

Some additional command line options recognized by \dvipdfmx\ are listed in
Table~\ref{TABLE:options}. In addition to this, the \code{-V} option for specifying
the output PDF version now accepts the version specification of a form \code{2.0}. Try
\begin{lstlisting}
dvipdfmx --help
\end{lstlisting}
for the list of command line options and their explanations.

\begin{table}
    \centering
    \begin{tabular}{lp{8cm}}\hline
        Option & Description \\ \hline\hline
        \code{-C} \textit{number} & Specify miscellaneous option flags. See,
        section of ``\hyperref[SEC:compatibility]{Incompatible Changes}'' for
        details. \\
        \code{-S} & Enable PDF encryption. \\
        \code{-K} \textit{number} & Set encryption key length. The default value
        is 40.\\
        \code{-P} \textit{number} & Set permission flags for PDF encryption.
        The \textit{number} is a 32-bit unsigned integer representing permission
        flags.
        See, section of ``\hyperref[SEC:encryption]{Encryption Support}''.
        The default value is \code{0x003C}.\\
        \code{-I} \textit{number} & Life of image cache in hours. By specifying
        value \code{0} \dvipdfmx\ erases cached images, and value \code{-1}
        erases all cached images and does not leave newly generated one. The
        default value is \code{-2}. (ignore image cache) \\
        \code{-M} & Process \MP\ generated PostScript file.\\
        \code{-E} & Always try to embed fonts \emph{regardless of
        licensing}.\\
        \code{-O} \textit{number} & Set maximum depth of open bookmark item.\\
        \hline
    \end{tabular}
    \caption{Additional command line options recognized by \dvipdfmx.}%
    \label{TABLE:options}
\end{table}


\section{Quick Guide}

\dvipdfmx\ is supposed to be used by users of \LaTeX\ packages for typesetting
CJK languages like H\LaTeX\ and CJK-\LaTeX, and \TeX\ variants such as \XeTeX,
p\TeX, and up\TeX.
This section is intended to be a quick guide for each users.

\subsection{\texorpdfstring{\XeTeX}{XeTeX}}

\XeTeX\ users
normally do not invoke the \dvipdfmx\ command directly. To control the
behavior of \dvipdfmx, please consider using the \code{dvipdfmx:config}
special explained in the section of ``\hyperref[SEC:specials]{Specials}.''
Some part of this document is irrelevant for \XeTeX\ users.

\subsection{p\TeX}
p\TeX\ users are at least required to install several auxiliary files
mentioned in the section of ``\hyperref[SEC:auxfiles]{Auxiliary Files}''
and to setup font-mappings. Just install the \package{adobemappings} and
\package{glyphlist} for auxiliary files. (As \TeX\ Live basic installation
requires them, they are probably already installed for \TeX\ Live users.)

For \TeX\ Live users, setting up fontmaps can be easily done with the help
of the \package{ptex-fontmaps} package and the \keyword{updmap} program.
To use with the IPAex fonts (contained in the \package{ipaex} package), for
example, run,
\begin{lstlisting}
kanji-config-updmap --sys ipaex
\end{lstlisting}
where the \option{--sys} option indicates the system-wide configuration.
After successful invocation of the above command, the IPAex fonts will
be used by default. The current setting can be checked via,
\begin{lstlisting}
kanji-config-updmap --sys status
\end{lstlisting}
For more information on the updmap program, try,
\begin{lstlisting}
kanji-config-updmap --help
\end{lstlisting}
or refer the documentation of the updmap program.

Alternatively, just for a quick test of installation, try the following:
\begin{lstlisting}
\documentclass{article}
\begin{document}
\special{pdf:mapline rml H KozMinProVI-Regular}
...Some Japanese text goes here...
\end{document}
\end{lstlisting}
In this example, PDF viewer which can handle substitute font is required since
\dvipdfmx\ does not embed fonts.

For using Japanese text in PDF document information and annotations, put
the following \code{special} command,
\begin{lstlisting}
\AtBeginDocument{\special{pdf:tounicode 90ms-RKSJ-UCS2}}
\end{lstlisting}
in the preamble.
The above \code{special} command instructs \dvipdfmx\ to convert text encoded
in Shift-JIS to Unicode. For EUC-JP, replace 90ms-RKJK-UCS2 with EUC-UCS2.

\subsection{up\TeX}
up\TeX\ users are basically the same as p\TeX\ users but there are two choices
for setting fontmaps. Setup fontmaps as mentioned above for p\TeX, or use
keyword \code{unicode} in the encoding field of the fontmap file.

The former might be easier as the auto-creation of fontmap files can be done
with the updmap program and the \package{ptex-fontmaps} package. But in
this method there are some difficulties when using fonts which employ character
collections (glyph repertoire) other than Adobe-Japan\lnum{1} in the case of PostScript
flavored OpenType fonts.
In the later case, the \package{adobemappings} package is not required
and newer PostScript flavored OpenType fonts which do not employ Adobe-Japan\lnum{1}
can be easily used too.

Using \code{unicode} is more simpler and intuitive thus it is recommended to
use this method.\footnote{For \TeX\ Live 2017. Earlier versions have buggy
support.}
A typical example fontmap entries for using Adobe's SouceHan series
might look like:
\begin{lstlisting}
urml    unicode  SourceHanSerifJP-Light.otf
urmlv   unicode  SourceHanSerifJP-Light.otf  -w 1
ugbm    unicode  SourceHanSansJP-Medium.otf
ugbmv   unicode  SourceHanSansJP-Medium.otf  -w 1
\end{lstlisting}

As in p\TeX, the following \code{special} instruction might be necessary for PDF
document information and annotations to be shown correctly:
\begin{lstlisting}
\AtBeginDocument{\special{pdf:tounicode UTF8-UCS2}}
\end{lstlisting}
Here, input encoding is assumed to be UTF-8.

\subsection{CJK-\LaTeX}

CJK-\LaTeX\ users are required to have \keyword{Subfont Definition Files}
to be installed. They are available as part of the \package{ttfutils} package.

To use TrueType Arphic fonts provided by the \package{arphic-ttf} package:
\begin{lstlisting}
\documentclass{article}
\usepackage{CJKutf8}
...Other packages loaded here...
\AtBeginDocument{%
  \special{pdf:tounicode UTF8-UCS2}%
  \special{pdf:mapline bsmiu@Unicode@ unicode bsmi00lp.ttf}%
 }
\begin{document}
\begin{CJK}{UTF8}{bsmi}
...some Chinese text goes here...
\end{CJK}
\end{document}
\end{lstlisting}
Here, \code{pdf:mapline} special is used to setup a font-mapping.

\section{Auxiliary Files}\label{SEC:auxfiles}

This section is mostly for supporting legacy encodings and legacy font format
such as PostScript Type1 font. \XeTeX\ users may skip this section.

\dvipdfmx\ has a capacity to handle various input encodings from 7-bit
encodings to variable-width multi-byte encodings. It also has some sort of
Unicode support. Several auxiliary files which are not common to \TeX\ users
are needed to enable those features. This section shortly describes about them.

\subsection{PostScript CMap Resources}

\keyword{PostScript CMap Resources}\footnote{See,
``\href{http://www.adobe.com/content/dam/Adobe/en/devnet/font/pdfs/5014.CIDFont\_Spec.pdf}%
{Adobe CMap and CIDFont Files Specification}''}
are required for supporting legacy encodings such as Shift-JIS, EUC-JP, \lnum{Big5},
and other East Asian encodings. \dvipdfmx\ internally identifies glyphs with
identifiers (CIDs\footnote{PostScript terminology ``Character IDentifier''.})
represented as an integer ranging from 0 to 65535 in the CID-based glyph access.
PostScript CMap Resources describes the mapping between sequences of input
character codes and CIDs. \dvipdfmx\ has an extensible support for multi-byte
encodings via PostScript CMap Resources.

CMap files for standard East Asian encodings, for use with Adobe's character
collections, are included in the \package{adobemapping} package.
The latest version of those CMap files maintained by Adobe can be found at
Adobe's GitHub Project page:\medskip

\url{http://github.com/adobe-type-tools/cmap-resources}
\medskip

Those files are mandatory for supporting p\TeX. up\TeX\ users may also
want to install them but they are not required.

\subsection{Subfont Definition Files}

CJK fonts usually contain several thousands of glyphs. For using such fonts
with (original) \TeX, which can only handle 8-bit encodings, it is necessary to
split a font into several ``subfonts''. Subfont Definition File (SFD)
specify the way how those fonts are split into subfonts. \dvipdfmx\ uses SFD
files to convert a set of subfonts back to a single font.

SFD files are not required for use with \TeX\ variants which can handle
multi-byte character encodings and large character sets such as p\TeX,
up\TeX,\XeTeX, and Omega.
H\LaTeX\ and CJK-\LaTeX\ users are required to have those files to be
installed. SFD files are available as a part of the \package{ttfutils} package
for \TeX\ Live users.

\subsection{The Adobe Glyph List and ToUnicode Mappings}

The Adobe Glyph List\footnote{See,
``\href{http://github.com/adobe-type-tools/agl-specification}{Adobe Glyph List Specification}''}
(AGL) describes correspondence between PostScript glyph names (e.g., \code{AE},
\code{Aacute},...) and Unicode character sequences representing them.
Some features described in the section ``Unicode Support'' requires AGL file.

\dvipdfmx\ looks for the file \code{glyphlist.txt} when conversion from
PostScript glyph names to Unicode sequences is necessary.
This conversion is done in various situations; when creating ToUnicode CMaps
for 8-bit encoding fonts, finding glyph descriptions from TrueType and OpenType
fonts when the font itself does not provide a mapping from PostScript glyph
names to glyph indices (version 2.0 ``post'' table), and when the encoding
\code{unicode} is specified for Type1 font.

The AGL file is included in the \package{glyphlist} package. The latest version
can be found at Adobe's GitHub site:\medskip

\url{http://github.com/adobe-type-tools/agl-aglfn}
\medskip

ToUnicode Mappings are similar to AGL but they describe correspondence
between CID numbers (instead of glyph names) and Unicode values.
The content of those files are the same as CMap Resources.
They are required when using TrueType fonts emulated as a CID-keyed font.
They should be found in the same directory as ordinary CMap files.

ToUnicode Mapping files are included in the \package{adobemapping} package.
Those files are not required for \XeTeX\ users.

\section{Overview of Extensions}

This section gives a quick overview of \dvipdfmx's extended capabilities.

\subsection{CJK Support}

There are many extensions made for supporting CJK languages. Features described
here is mainly for CJK languages but their use is actually not limited to it.
Those features are implemented in a generic way so that it can be beneficial to
users who are not involved in CJK languages.

\subsubsection{Legacy Multi-byte Encodings}

\dvipdfmx\ has an extensible support for encodings by means of
PostScript CMap Resources. Just like \code{enc} files are written for 8-bit
encodings, one can write their own CMap files to support custom encodings.
See, Adobe's technical notes\cite{ADOBE} for details on PostScript CMap Resources.

\subsubsection{Vertical Writing}

\dvipdfmx\ supports the vertical writing extension used by p\TeX\ and up\TeX.
A DVI instruction to set the writing mode is supported. The OpenType Layout
GSUB Feature is supported for selecting vertical version of glyphs.

\begin{figure}
\centering
\jphoritext\hspace{24pt}\jpverttext%
\caption{An example of horizontal and vertical text;
left and right corner brackets are replaced with their vertical counterparts.}%
\label{FIG:verttext}
\end{figure}

\subsection{Unicode Support}

Unicode support here consists of two parts: Supporting Unicode as an input
encoding and making output PDF files ``Unicode aware''.

\subsubsection{Unicode as Input Encoding}

\dvipdfmx\ recognizes an additional keyword \code{unicode} in fontmap files
to declare that Unicode values are used in the input DVI file. Unicode support is
basically limited to the Basic Multilingual Plane (BMP) since there are no
support for code ranges that requires more than three bytes in TFM and extended
TFM formats.

\subsubsection{ToUnicode CMap Support}

In PDF, it is often the case that text is not encoded in Unicode.
However, modern applications usually want them represented in Unicode to make
it usable as text information.
The ToUnicode CMap is a bridge between PDF text string encodings and Unicode
encodings, and makes it possible to extract text in PDF files as
Unicode encoded strings. It is important to make resulting PDF search-able and
copy-and-past-able. Dvipdfmx supports auto-creation of ToUnicode CMaps.

It will not work properly for multiply encoded glyphs due to fundamental
limitations of Unicode conversion mechanism with ToUnicode CMaps.

\subsection{Other Extensions}

\dvipdfmx\ can generate encrypted PDF documents to protect its contents from
unauthorized access. It is limited to password-based authentication, and
public-key based authentication is not supported. The 256-bit AES encryption is
also supported for PDF version 1.7 and 2.0 setting although it may not be supported
by PDF viewers.

There are various other improvements over \dvipdfm. The most notable one is
more improved PDF input and output support: The cross-reference stream and
object stream introduced in \lnum{PDF-1.5} are also supported.


\chapter{Graphics}\label{SEC:graphics}

\section{Image Inclusion}

The basics of incorporating images into output PDF is the same as in \dvipdfm.
To do this, \LaTeX\ users can simply use the \package{graphicx} package.
(possibly with the driver option \code{dvipdfmx})
This section is \emph{not} for providing a how-to guide to include images but
just for supported graphics and image formats along with supported features.

Graphics support was mostly rewritten in \dvipdfmx.
Support for BMP and JPEG\lnum{2000} was added. An effort to preserve more
information originally found in included images, e.g., embedded ICC Profiles
and XMP Metadata, was made.

However, \dvipdfmx\ does not support various features common to graphics
manipulation programs such as resampling, color conversion, and selection of
compression filters. Thus, it is recommended to use other programs specialized
in image manipulation for preparation of images.

\subsection{Supported Graphics File Formats}

Supported formats are, PNG, JPEG, \lnum{JPEG2000}, BMP, PDF, and \MP\ generated
EPS. All other format images, such as SVG and PostScript, must be converted to
PDF before inclusion. The \option{-D} option, as in \dvipdfm, can be used for
filtering images.

\subsubsection{Notes on PNG Support}

PNG is supported as in \dvipdfm\ with many improvements.

PNG support includes most of important features of PNG format such as color
palette, transparency, 16-bit bit-depth color, embedded ICC Profiles,
calibrated color, and embedded XMP Metadata.

In including PNG images, \dvipdfmx\ first decompresses image data and then
compresses (if requested) it again.
For better compression ratio, a preprocessing filter (Predictor filter) might
be applied before compression.
\dvipdfmx\ supports the TIFF Predictor 2 and the PNG optimum filter.
However, there is yet no way to specify which predictor function is to be used
and currently PNG optimum filter is always employed.

Predictor filter is a preprocessing filter to image data for improving compression.
Using a predictor filter usually gives better compression
but in many cases compression speed becomes significantly slower.
Try \option{-C 0x20} command line option to disable predictor filters and to
check if slowness is due to filtering.

For the PNG optimum filter, a heuristic approach, ``minimum sum of absolute
differences'', is employed for finding the most optimal filter to be used.
See, discussion in the PNG Specification ''Filter selection'':\medskip

\url{http://www.w3.org/TR/2003/REC-PNG-20031110/\#12Filter-selection}
\medskip

As built-in support for the sRGB color space is absent in PDF,
the sRGB color can only be represented precisely by means of the sRGB ICC Profile.
However, for sRGB color PNG images, \dvipdfmx\ uses an approximate calibrated
RGB color space instead.
For approximating the sRGB color, the gamma and CIE \lnum{1931} chromaticity values
mentioned in the PNG Specification is used.
See, the following page for more information:\medskip

\url{http://www.w3.org/TR/2003/REC-PNG-20031110/\#11sRGB}
\medskip

\dvipdfmx\ also supports calibrated color with the \code{gAMA} and the \code{cHRM} chunk.
These chunks carry information on more accurate color representation. 
Some software programs, however, write only \code{cHRM} but do not record the gamma value
although the PNG specification recommends to do so. Gamma value 2.2 is assumed if only
\code{cHRM} is present but \code{gAMA} is not.

Some PNG features are unavailable depending on output PDF version setting. Please refer
Table~\ref{TABLE:PNGfeat} for more details.

\begin{table}
  \centering
  \begin{tabular}{lp{8cm}}\hline
      Feature & PDF Version Required \\ \hline\hline
      16-bit Color Depth & Version 1.5 \\
      Transparency & Full support for alpha channel requires PDF version 1.4.
        Color key masking (a specific color is treated as fully transparent)
        requires 1.3.\\
      XMP Metadata & Version 1.4 \\
      ICC Profile & Version 1.3 \\
      \hline
  \end{tabular}
  \caption{PNG features and corresponding PDF versions required.}%
  \label{TABLE:PNGfeat}
\end{table}

\subsubsection{JPEG and \lnum{JPEG2000}}

JPEG format is supported as in \dvipdfm. In addition to this, \lnum{JPEG2000}
is also supported.

JPEG and \lnum{JPEG2000} image inclusion is basically done as ''pass-through'',
that is, image data is not decompressed before inclusion. So, although these
formats are basically lossy, there should be no quality loss when including
images.

JPEG is relatively well supported. \dvipdfmx\ supports embedded ICC Profiles
and CMYK color. Embedded XMP metadata is also preserved in the output PDF.
JFIF or Exif data might be used to determine image's physical size.

As the PDF specification does not require information irrelevant to
displaying images to be embedded, \dvipdfmx\ does not embed whole data.
Especially, not all application specific data is retained. Application 
specific data such as JFIF, Exif, and \code{APP14} Adobe marker are
preserved.
Please note that XMP and Exif data which may contain location information
where the photo was taken is always preserved in the output PDF file.

\lnum{JPEG2000} is also supported. It is restricted to JP2 and JPX baseline
subset as required by the PDF specification. It is not well supported and still in
an experimental stage. J2C format and transparency are not supported.

\subsubsection{PDF Support}

PDF inclusion is supported as in \dvipdfm, with various important enhancement
over \dvipdfm\ for more robust inclusion. \dvipdfmx\ can handle cross-reference
streams and object streams introduced in \lnum{PDF-1.5}.
\dvipdfmx\ also supports inclusion of PDF pages other than the first page.
However, tagged PDF may cause problems and annotations are not kept.

As there are no clear way to determine the natural extent of the graphics content
to be clipped, \dvipdfmx\ preferably try to find the \emph{crop box} to decide
image size. If there are no crop box \emph{explicitly} specified,%
\footnote{There are some accusations by Japanese \TeX\ users as
''violating the PDF spec.'' regarding this point. However, what we are
talking about is how to guess the \emph{natural} or \emph{intended} size of
images but not the default value of the PDF crop box itself.} then it
tries to refer other boundary boxes such as the \emph{art box}. If there are no
possible boundaries of the graphics content explicitly specified, the \emph{media box},
which is the boundaries of the physical medium on which the page is to be printed,
is used as the last resort.

The \code{pdf:image} special supports additional keys, ``\code{page}'' and
``\code{pagebox}''. The \code{page} key takes an integer value representing
the page number of PDF page to be included, and the \code{pagebox} takes one
of the keywords \code{mediabox}, \code{cropbox}, \code{artbox}, \code{bleedbox},
or \code{trimbox} for selecting page's boundary box to be used. For example,
\begin{lstlisting}
\special{pdf:image pagebox artbox page 3 (foo.pdf)}
\end{lstlisting}
includes 3rd page of `foo.pdf' with the boundary box set to the art box.

\subsubsection{Other Image Formats}

For \MP\ generated Encapsulated PostScript (EPS) files, multi-byte encoding
support is added.
\dvipdfmx\ also supports ``\MP\ mode''. When \dvipdfmx\ is invoked with
\option{-M} option, it enters in \MP\ mode and processes a \MP\
generated EPS file as an input.%
\footnote{\code{prologue} should be set to \code{2}.}

BMP support is also added. It is limited to uncompressed or RLE-compressed
raster images. Extensions are not (won't be) supported.

For image formats not natively supported, the \code{-D} option can be
used to convert images to PDF format before inclusion, as in \dvipdfm.
In \dvipdfmx, the letter \code{v} in the \code{-D} option argument is expanded
to the output PDF version.

\subsection{Image Cache}

Caching of images generated via filtering command specified with \option{-D}
option is supported. It solves the problems that image inclusion becomes very
slow when external filtering program such as GhostScript is invoked each time
images are included.

Use \option{-I} option to enable this feature:
\begin{lstlisting}
-I 24
\end{lstlisting}
where the integer represents the life of cache files, 24 hours in this example.
Zero and negative values have a special meaning.
Value 0 for ``erase old cached images while leaving newly created one'',
-1 for ``erase all cached images'', and -2 for ``ignore image cache''.
Default value is set to -2.

\section{Graphics Drawing}

\dvipdfmx\ does not offer a high level interface to draw graphics objects.
A possible way to draw graphics is to write raw PDF graphics drawing codes and
then to insert them into the output via \code{special} commands.

To show an example, the following code:
\begin{lstlisting}
\special{pdf:content
  1 0 0 1 0 0 cm
  0 100 m
  80 100 120 80 120 0 c
  S
}
\end{lstlisting}
draws a Bézier curve,
\begin{center}
\specialbox{120bp}{100bp}{pdf:content 1 0 0 1 0 0 cm 0 100 m 80 100 120 80 120 0 c S}
\end{center}
The \code{pdf:content} special is used here which is useful for inserting an
isolated graphics object.

The above example illustrates a typical example of PDF graphics drawing.
It consists of three parts; setting graphics state, constructing a path, and painting
a path. A Graphic object are specified as a sequence of operators and their operands
using \emph{postfix notation}. \keyword{Graphics state operators} comes first,
\code{cm} in this example sets the current transformation matrix (CTM). Then,
\keyword{path construction} operators follow; move to position $(0, 100)$,
append a Bézier curve from current point to $(120, 0)$ with control points
$(80, 100)$ and $(120, 80)$. Finally, a \keyword{path painting} operator
comes to draw the constructed path.
In this example the stroking operator \code{S}
is used.

\subsection{The \code{pdf:content} Special}

The \code{pdf:content} special can be used for drawing an \emph{isolated}
graphics object. It sets the origin of graphics drawing operators supplied to
this command to the position where it is inserted.
The whole content is enclosed by a pair of graphics state save-restore
operators. So for example, a color change made within a \code{pdf:content}
command takes an effect only within the content of this special.

\subsection{Guide to PDF Graphics Operators}

PDF employs essentially the same imaging model as PostScript.
So, it is easy to learn about PDF graphics drawing for people who are
well accustomed to PostScript. This section is intended to be a short guide
for PDF graphics operators.

\subsubsection{Graphics State Operators}

The \code{cm} operator modifies CTM by concatenating the specified matrix.
Operands given to this operators are six numbers each representing
transformation matrix elements:
translation represented as $[1, 0, 0, 1, t_x, t_y]$,
scaling $[s_x, 0, 0, s_y, 0, 0]$,
rotation $[\cos\theta, \sin\theta, -\sin\theta, \cos\theta, 0, 0]$.

To uniformly scale the object, just use
\begin{lstlisting}
2.0 0 0 2.0 0 0 cm
\end{lstlisting}

The \code{w} operator sets the line width, e.g., `\code{2 w}' sets the line
width to 2. Here is an example of drawing a rotated rectangle:
\begin{lstlisting}
0.866 0.5 -0.5 0.866 30 2 cm 5 w 0 0 100 50 re S
\end{lstlisting}
\begin{center}
\specialbox{120bp}{96bp}{pdf:content 0.866 0.5 -0.5 0.866 30 2 cm 5 w 0 0 100 50 re S}%
\end{center}

Transformations can be sequentially applied; for the above example,
\begin{lstlisting}
1 0 0 1 30 2 cm 0.866 0.5 -0.5 0.866 0 0 cm
5 w 0 0 100 50 re S
\end{lstlisting}
gives the same result.

To specify colors, use \code{RG}, \code{rg}, \code{K}, and \code{k} operators,
for RGB or CMYK color for stroking (upper-case) and nonstroking (lower-case).
\begin{lstlisting}
0.866 0.5 -0.5 0.866 30 2 cm 5 w
1 0.4 0 0 K 1 0 0 0 k
0 0 100 50 re B
\end{lstlisting}
where the \code{B} operator fill and then stroke the path.
\begin{center}
\specialbox{120bp}{96bp}{pdf:content 0.866 0.5 -0.5 0.866 30 2 cm 5 w
1 0.4 0 0 K 1 0 0 0 k 0 0 100 50 re B}
\end{center}

A dash pattern can be specified with the \code{d} operator.
Operands for this operator are the \keyword{dash array} and
the \keyword{dash phase}.
For example, to specify a dash pattern with 6-on 4-off starting with phase 0:
\begin{lstlisting}
[6 4] 0 d 2 w 0 0 m 320 0 l S
\end{lstlisting}
draws the following dashed line:
\begin{center}
\specialbox{320bp}{30bp}{pdf:content
1 0 0 1 0 20 cm [6 4] 0 d 2 w 0 0 m 320 0 l S}
\end{center}

Other important operators are \code{q} and \code{Q}, which saves and restores
the graphics state.
\begin{lstlisting}
1 0 0 1 30 2 cm
q
0.866 0.5 -0.5 0.866 0 0 cm
[6 4] 0 d 2 w 0 0 100 50 re S
Q
-30 0 m 90 0 l S
0 -2 m 0 96 l S
\end{lstlisting}
In the above example, \code{d}, \code{w}, and rotation only take effect within
the \code{q}-\code{Q} block. The portion drawing two straight lines is unaffected
by these graphics state change.
\begin{center}
\specialbox{120bp}{96bp}{pdf:content
1 0 0 1 30 2 cm
q
0.866 0.5 -0.5 0.866 0 0 cm
[6 4] 0 d 2 w 0 0 100 50 re S
Q
-30 0 m 90 0 l S
0 -2 m 0 96 l S
}
\end{center}

For a (incomplete) list of graphics state operators, see
Talbe~\ref{TAB:operatorsGS}.

\begin{table}
    \centering
    \begin{tabular}{llp{7.6cm}}\hline
        Operands & Operator & Description\\ \hline\hline
        --- & \code{q} & Save the current graphics state.\\
        --- & \code{Q} & Restore the previously saved graphics state.\\
        $a$ $b$ $c$ $d$ $e$ $f$ & \code{cm} & Modify the current transformation matrix
        by concatenating the specified matrix.\\
        \textit{width} & \code{w} & Set the line width.\\
        \textit{array} \textit{phase} & \code{d} & Set the line dash pattern.\\
        $r$ $g$ $b$ & \code{RG} & Set the stroking color space to RGB and
        set the stroking color as specified.\\
        $r$ $g$ $b$ & \code{rg} & Set the nonstroking color space to RGB and
        set the nonstroking color as specified.\\
        $c$ $m$ $y$ $k$ & \code{K} & Set the stroking color space to CMYK and
        set the stroking color as specified.\\
        $c$ $m$ $y$ $k$ & \code{k} & Set the nonstroking color space to CMYK and
        set the nonstroking color as specified.\\
        \hline
    \end{tabular}
    \caption{A few examples of graphics state operators and color operators.}%
    \label{TAB:operatorsGS}
\end{table}

\subsubsection{Path Construction Operators}

A path construction normally starts with a \code{m} operator which moves the
current point to the specified position and then sequences of other path
construction operators follow. The path currently under construction is
called the \keyword{current path}. A sequence of path construction operators
appends segments of path to the current path and then move the
\keyword{current point} to the end point of appended path.
A typical sequence of path construction looks like,
\begin{lstlisting}
100 50 m
100 78 78 100 50 100 c
22 100 0 78 0 50 c
0 22 22 0 50 0 c
78 0 100 22 100 50 c
S
\end{lstlisting}
\begin{center}
\specialbox{120bp}{80bp}{pdf:content
100 50 m
100 78 78 100 50 100 c
22 100 0 78 0 50 c
0 22 22 0 50 0 c
78 0 100 22 100 50 c
S}
\end{center}
This example is an approximated circle drawn by four Bézier curves.

Table~\ref{TAB:operators} shows a list of path construction operators.
Those who are accustomed to the PostScript language should note that in PDF
the current path is not a part of the graphics state,
and hence is \emph{not} saved and restored along with the other graphics state
parameters.

\begin{table}
    \centering
    \begin{tabular}{llp{6.9cm}}\hline
      Operands & Operator & Description \\ \hline\hline
      $x$ $y$ & \code{m} & Begin a new path by moving the current point specified by given operands.\\
      $x$ $y$ & \code{l} & Append a line segment from the current point to the point specified.\\
      $x_1$ $y_1$ $x_2$ $y_2$ $x_3$ $y_3$ & \code{c} & Append a Bézier curve to the current path.
        Two Control points and the end point given as operands.\\
      $x_2$ $y_2$ $x_3$ $y_3$ & \code{v} & Append a Bézier curve to the current path. Using the current
        point and first two operand as the Bézier control points.\\
      $x_1$ $y_1$ $x_3$ $y_3$ & \code{y} & Append a Bézier curve to the current path. The second
      control point coincides with the end point.\\
      --- & \code{h} & Close the current path by appending a straight line segment from the current point
        to the starting point of the path.\\
      $x$ $y$ \textit{width} \textit{height} & \code{re} & Append a rectangle. First two operands for the
        position of lower-left corner, third and forth operand representing width and
      height.\\
      \hline
    \end{tabular}
    \caption{List of path construction operators. All operators move the current point to the end
      point of appended path.}\label{TAB:operators}
\end{table}

\subsubsection{Path Painting Operators}

There are basically four kind of path painting operators: \code{S}, \code{f},
\code{B}, and \code{n}. The first three for ``stroke'', ``fill'', and
``fill then stroke'' operators respectively, and the last one \code{n}
paints nothing but end the path object. For filling operators, there are
two kind of operators depending on how ``insideness'' of points are determined:
the \keyword{non-zero winding number rule} and the \keyword{even-odd rule}.
The even-odd rule operators are \code{f*} and \code{B*}.

The following example illustrates the difference:
\begin{lstlisting}
0 0 100 100 re 20 20 60 60 re f
1 0 0 1 120 0 cm
0 0 100 100 re 20 20 60 60 re f*
\end{lstlisting}
\begin{center}
\specialbox{220bp}{100bp}{pdf:content
0 0 100 100 re 20 20 60 60 re f
1 0 0 1 120 0 cm
0 0 100 100 re 20 20 60 60 re f*}
\end{center}
The ``interior'' of the ``inner'' square has a non-zero even winding number.
(In this example, counter-clockwise direction is assumed for both of
two \code{re} operators.)

\chapter{Specials}

\section{PDF Specials}\label{SEC:specials}

\dvipdfmx\ recognizes various special commands originally introduced in
\dvipdfm. Please refer to the ``Dvipdfm User's Manual''\cite{DVIPDFM} for detailed
information on PDF specials.

\subsection{Additions to PDF Specials}

Several \code{special} commands are added for more flexible PDF generation:
creation of arbitrary stream objects, controlling \dvipdfmx\ behavior, and some
specials which might be useful for graphics drawing.

\subsubsection{PDF Object Manipulation}

PDF object manipulation is a key feature of PDF specials.
The \code{pdf:fstream} special is added in \dvipdfmx\ which enables creation of
PDF stream object from an existing file. The syntax of this special is,
\begin{lstlisting}
pdf:fstream @identifier (filename) <<dictionary>>
\end{lstlisting}
where identifier and filename (specified as a PDF string object) are
mandatory and a dictionary object, following the filename, which is to be added
to the stream dictionary is optional.

For example, to incorporate XMP Metadata stored from the file \code{test.xmp},
\begin{lstlisting}
\special{pdf:fstream @xmp (test.xmp) <<
    /Type    /Metadata
    /Subtype /XML
>>}
\special{pdf:put @catalog << /Metadata @xmp >>}
\end{lstlisting}

Similarly, \code{pdf:stream} special can be used to create a PDF stream
object from a PDF string instead of a file.
\begin{lstlisting}
pdf:stream @identifier (stream contents) <<dictionary>>
\end{lstlisting}

This special might be useful for creating a simple image inline.
\begin{lstlisting}
\special{pdf:stream @myim01
  <5500AAC05500AAC05500AAC05500AAC05500>
  <<
    /Type /XObject
    /Subtype /Image
    /BitsPerComponent 1
    /ColorSpace /DeviceGray
    /Width 9
    /Height 9
  >>
}
\special{pdf:put @resources <<
  /XObject << /MyIM01 @myim01 >>
>>}
\special{pdf:content 81 0 0 81 0 0 cm /MyIM01 Do}
\end{lstlisting}
The above example draws an image like Figure~\ref{FIG:stream}.
\begin{figure}
  \centering
  \makebox[81bp][l]{\raisebox{-81bp}[0bp][81bp]{%
  \special{pdf:stream @myim01 <5500AAC05500AAC05500AAC05500AAC05500>
  <<
    /Type /XObject
    /Subtype /Image
    /BitsPerComponent 1
    /ColorSpace /DeviceGray
    /Width  9
    /Height 9 
  >>}
  \special{pdf:put @resources <<
    /XObject << /MyIM01 @myim01 >>
  >>}
  \special{pdf:content 81 0 0 81 0 0 cm /MyIM01 Do}
}}
\caption{An image created by \code{pdf:stream} special.}%
\label{FIG:stream}
\end{figure}

\subsubsection{Controlling Font Mappings}

\code{pdf:mapline} and \code{pdf:mapfile} specials can be used to append a
fontmap entry or to load a fontmap file:
\begin{lstlisting}
pdf:mapline foo unicode bar
pdf:mapfile foo.map
\end{lstlisting}

\subsubsection{Specifying The Output PDF Version}

\code{pdf:majorversion} and \code{pdf:minorversion} specials can be used to
specify major and minor version of output PDF.
\begin{lstlisting}
pdf:minorversion 3
\end{lstlisting}

\subsubsection{Encryption}

To protect output PDF with encryption, use \code{pdf:encrypt} special
\begin{lstlisting}
pdf:encrypt userpw (foo) ownerpw (bar) length 128 perm 20
\end{lstlisting}
where user-password (\code{userpw}) and owner-password (\code{ownerpw}) must be
specified as PDF string objects. (which can be empty)
Numbers specifying key-length and permission flags here are decimal numbers.
See, section ``\hyperref[SEC:encryption]{Encryption Support}'' for a brief
description of permission flags.

\subsubsection{PDF Document Creation}

\newfeature{Addition in \TeX\ Live 2019}

As a convenience, the \code{pageresources} special is added, which puts given page
resources into subsequent page's \keyword{Resource Dictionary}. For example,
\begin{lstlisting}
\special{pdf:pageresources <<
  /ExtGState << /MyGS01 << /ca 0.5 /CA 0.5 >> >>
>>}
\end{lstlisting}
puts an ExtGState resource named \code{MyGS01} into the current page's and all
subsequent pages' resource dictionary.

Other notable extensions are \code{code}, \code{bcontent}, and
\code{econtent}. The \code{code} special can be used to insert raw PDF graphics
instructions into the output. It is different from \dvipdfm's
\code{content} special in that it does not enclose contents with a \code{q}
and \code{Q} (save-restore of graphics state) pair.
A typical usage of this special is:
\begin{lstlisting}
\special{pdf:code q 1 Tr}
...some text goes here...
\special{pdf:code Q}
\end{lstlisting}
which changes text rendering mode to 1, as shown in Figure~\ref{FIG:trmode}.

\begin{figure}
\centering
\mbox{\special{pdf:code q 1 w 1 Tr}%
{\fontsize{200pt}{200pt}\selectfont\textchi}%
\special{pdf:code Q}}%
\caption{A character drawn in the PDF text rendering mode 1.}%
\label{FIG:trmode}
\end{figure}

Be careful on using this special as it is very easy to generate
broken PDF files. The \code{bcontent} and \code{econtent} pair is somewhat
fragile and might be incompatible to other groups of special commands.
It may not always be guaranteed to work as `expected'.

\subsection{ToUnicode Special}

PDF allows users to attach various additional information such as document information,
annotation, and navigation information (like bookmarks) to their document.
All human-readable text, \keyword{text string}, contained in such information must
be encoded either in \keyword{PDFDocEncoding} or UTF-16BE with Unicode byte order
marker. However, as many users can't prepare their document with text strings
properly encoded, there is a special kind of special command, \code{pdf:tounicode},
which can be used to convert text strings into the appropriate Unicode form.
Note that this feature is provided just as a remedy for users incapable of encoding
text strings properly.

For example,
\begin{lstlisting}
\special{pdf:tounicode 90ms-RKSJ-UCS2}
\end{lstlisting}
declares that \emph{some} of text strings should be re-encoded according to specified
conversion CMap \code{90ms-RKSJ-UCS2}.

Conversion is done only for arguments to several PDF special commands such as
\code{docinfo}, \code{ann}, and \code{out} but not for that of general object creation
specials. Note that not all PDF string objects are subject to this conversion.
By default, only dictionary entries listed below are converted.
\begin{lstlisting}
Title Author Subject Keywords Creator Producer Contents Subj
TU T TM 
\end{lstlisting}

\newfeature{Addition in \TeX\ Live 2019}

The list of dictionary entries subject to conversion can be extended by supplying
additional dictionary keys as an array object:
\begin{lstlisting}
  \special{pdf:tounicode 90ms-RKSJ-UCS2 [/RC /DS]}
\end{lstlisting} 

If the name of conversion CMap contains one of the keywords RKSJ, \lnum{B5}, GBK, and KSC,
PDF string objects are treated specially when they are parsed. A two byte sequence
starting with the first byte's high bit set is treated ''as is'' so that
the \code{0x5c} byte appears in the second byte is not treated as an escape sequence.
This behavior is not compliant to the PDF specification.

\subsection{PDF Special Examples}

This section shows several examples of special command usage.
It is intended to be a hint for advanced users, so uninterested users can safely skip
this section. In many cases, using \dvipdfmx\ PDF specials requires knowledge on PDF.
Please refer to Adobe's ''PDF Reference''\cite{ADOBE}.

\subsubsection{Annotations}

In this section, some useful special commands for creating \keyword{annotations}
are explained. Note that viewer support is required for annotations to be
displayed correctly.

First start with a very simple \keyword{Text Annotation} for attaching a comment.
This feature is supported by many PDF viewer applications.
\marginnote{\special{pdf:ann width 20bp height 20bp
  <<
     /Type     /Annot
     /Subtype  /Text
     /Name     /Comment
     /T        (Text Annotation Example)
     /Subj     (An Example of Text Annotations)
     /Contents (A Quick Brown Fox Jumped Over The Lazy Dog.)
  >>
}}

\begin{lstlisting}
\special{pdf:ann width 20bp height 20bp
  <<
     /Type     /Annot
     /Subtype  /Text
     /Name     /Comment
     /T        (Text Annotation Example)
     /Subj     (An Example of Text Annotations)
     /Contents (A Quick Brown Fox Jumped Over The Lazy Dog.)
  >>
}
\end{lstlisting}
\code{pdf:ann} special is used to create an annotation. A small icon shall
be shown on the side margin. Here, dictionary entry \code{T} is for the tilte,
\code{Subj} for the subject of this annotation, and \code{Contents} for
the text string to be shown in the body of this annotation.

Likewise, \keyword{Rubber Stamp Annotation}, which places a rubber stamp
like figure,
\begin{center}
\specialbox{150bp}{50bp}{pdf:ann width 150bp height 50bp
  <<
     /Type     /Annot
     /Subtype  /Stamp
     /Name     /Approved
  >>
}
\end{center}

\begin{lstlisting}
\special{pdf:ann width 150bp height 50bp
  <<
     /Type     /Annot
     /Subtype  /Stamp
     /Name     /Approved
  >>
}
\end{lstlisting}
Other keywords such as \code{Expired}, \code{Final}, \code{Draft}, and so on,
can be used in place of \code{Approved}.

One can create stamps of their own style. For this purpose, other specials
\code{pdf:bxobj} and \code{pdf:exobj} can be used for designing stamps.
Those specials ``capture'' all typeset material enclosed by them into a PDF
\keyword{Form XObject}, which is a reusable graphics object like included images.

For a simple example,
\begin{lstlisting}
\special{pdf:bxobj @MyStamp
             width 280pt height 0pt depth 40pt}
\addfontfeature{Scale=4,Color=FF9933}My Own Stamp
\special{pdf:exobj}
\end{lstlisting}
It captures typeset material ``My Own Stamp'' (this example uses \code{fontspec}
package's command for changing font size and text color) into the object labeled as
\code{MyStamp} for later reuse.
Then, \code{AP} (\keyword{appearance dictionary}) entry for controlling the appearance
of annotations is used as,
\begin{lstlisting}
\special{pdf:ann width 280pt height 40pt
  <<
    /Type    /Annot
    /Subtype /Stamp
    /AP      << /N @MyStamp >>
  >>
}
\end{lstlisting}
The image captured into the object \code{MyStamp} is used as ``Normal''
(\code{AP} dictionary entry \code{N}) appearance.
(\code{R} for ``Rollover'' and \code{D} for ``Down'' can be used.)

\parbox[t][0pt][c]{280pt}{%
\special{pdf:bxobj @MyStamp width 280pt height 0pt depth 40pt}%
\addfontfeature{Scale=4,Color=FF9933}My Own Stamp%
\special{pdf:exobj}%
}%

The result is:
\begin{center}
\specialbox{280bp}{40bp}{pdf:ann width 280pt height 40pt
  <<
    /Type    /Annot
    /Subtype /Stamp
    /AP      << /N @MyStamp >>
  >>
}
\end{center}

With the following code, \dvipdfmx\ reads the source file and creates a stream
object named \code{SourceFile}, and then creates file attachment annotation.
\marginnote{%
\special{pdf:fstream @SourceFile (\jobname.tex)}%
\special{pdf:ann width 10bp height 20bp
   <<
      /Type /Annot
      /Subtype /FileAttachment
      /FS <<
            /Type /Filespec
            /F    (\jobname.tex)
            /EF   << /F @SourceFile >>
          >>
      /Name     /PushPin
      /C        [0.8 0.2 0.2]
      /T        (The Dvipdfmx User's Manual)
      /Subj     (The Dvipdmfx User's Manual)
      /Contents (XeLaTeX source file of this manual.)
   >>
}}%
\begin{lstlisting}
\special{pdf:fstream @SourceFile (\jobname.tex)}%
\special{pdf:ann width 10bp height 20bp
   <<
      /Type /Annot
      /Subtype /FileAttachment
      /FS <<
            /Type /Filespec
            /F    (\jobname.tex)
            /EF   << /F @SourceFile >>
          >>
      /Name     /PushPin
      /C        [0.8 0.2 0.2]
      /T        (The Dvipdfmx User's Manual)
      /Subj     (The Dvipdfmx User's Manual)
      /Contents (XeLaTeX source file of the manual.)
   >>
}
\end{lstlisting}
A push-pin image must be shown on the margin if viewer supports this kind of
annotation.
PDF viewer applications are required to provide predefined icon appearances
at least for the following standard icons: \code{Graph}, \code{PushPin},
\code{PaperClip}, and \code{Tag}.

\subsubsection{Special Color Space}

This section shows various examples of using \keyword{Special color spaces}.
Examples in this section have a common structure. They consist of essentially
three parts. The first part is for defining color space itself.
PDF object creation commands like \code{pdf:obj} and \code{pdf:stream} are used
for this purpose. Next is for registering color space resources in the page's
\keyword{Resource Dictionary}. It can be done via \code{pdf:put} command as,
\begin{lstlisting}
\special{pdf:put @resource <<
  /Category << ...key-value pairs... >>
>>}
\end{lstlisting}
where \code{@resource} is a special keyword representing current page's
Resource Dictionary and \code{Category} (to be replaced by actual category
name) is a category name such as \code{ColorSpace}.
Finally, graphics objects are placed, with or with a combination of
text and, PDF drawing operators inserted by the \code{pdf:code} or the
\code{pdf:contents} special.

%%EXAMPLE: Separation Color Space
The first example is the \keyword{Separation} color space:
\special{pdf:stream @TintTransform1
  ({0 exch dup 0.62 mul exch 0})
  <<  /FunctionType 4
      /Domain [0.0 1.0]
      /Range  [0.0 1.0 0.0 1.0 0.0 1.0 0.0 1.0]
  >>
}%
\special{pdf:stream @TintTransform2
  ({dup 0.78 mul exch dup 0.05 mul exch 0.71 mul 0})
  <<  /FunctionType 4
      /Domain [0.0 1.0]
      /Range  [0.0 1.0 0.0 1.0 0.0 1.0 0.0 1.0]
  >>
}%
\special{pdf:obj @Orange [
    /Separation /Orange /DeviceCMYK @TintTransform1
  ]
}%
\special{pdf:obj @Green [
    /Separation /Green /DeviceCMYK @TintTransform2
  ]
}%
\begin{center}
\mbox{%
\special{pdf:put @resources <<
    /ColorSpace << /CS01 @Orange /CS02 @Green >>
  >>
}%
\fontsize{40pt}{40pt}\selectfont
\special{pdf:code q /CS01 cs 1.0 scn}Orange\special{pdf:code Q} and
\special{pdf:code q /CS02 cs 1.0 scn}Green\special{pdf:code Q}
}
\end{center}
\begin{lstlisting}
\special{pdf:stream @TintTransform1
  ({0 exch dup 0.62 mul exch 0})
  <<  /FunctionType 4
      /Domain [ 0.0 1.0 ]
      /Range  [ 0.0 1.0 0.0 1.0 0.0 1.0 0.0 1.0 ]
  >>
}
\special{pdf:stream @TintTransform2
  ({dup 0.78 mul exch dup 0.05 mul exch 0.71 mul 0})
  <<  /FunctionType 4
      /Domain [ 0.0 1.0 ]
      /Range  [ 0.0 1.0 0.0 1.0 0.0 1.0 0.0 1.0 ]
  >>
}
\special{pdf:obj @Orange [
    /Separation /Orange /DeviceCMYK @TintTransform1
  ]
}
\special{pdf:obj @Green [
    /Separation /Green /DeviceCMYK @TintTransform2
  ]
}
\mbox{%
  \special{pdf:put @resources <<
      /ColorSpace << /CS01 @Orange /CS02 @Green >>
    >>
  }%
  \special{pdf:code q /CS01 cs 1.0 scn}
  Orange
  \special{pdf:code Q}
  and
  \special{pdf:code q /CS02 cs 1.0 scn}
  Green
  \special{pdf:code Q}
}
\end{lstlisting}

\code{TintTransform}'s defined here are functions for transforming
\keyword{tint} values into approximate colors in the
\keyword{alternate color space}
(\code{DeviceCMYK} in this example). PostScript calculator functions are used
for converting a single component value representing ``Orange'' or ``Green''
into four component CMYK values to approximate those colors. The ``Orange''
color $v$ is approximated as $(0, 0.62v, v, 0)$ in CMYK color space for
alternate display here.

The \code{cs} operator for selecting color space and the \code{scn} operator
for color values are used in the \code{pdf:code} special.
Be sure that the \code{pdf:put} command, which puts color space resources
into the current page's Resource Dictionary, goes into the same page as
subsequent drawing commands.

\dvipdfmx\ currently does not have an easy interface for using various color
space families such as CIE-Based color spaces (e.g., calibrated colors and
color space with an ICC profile) and Special color spaces (e.g., indexed,
separation, shading and patterns).

%%EXAMPLE: Shading
Another example is a \keyword{shading pattern}:
\begin{lstlisting}
\special{pdf:put @resources <<
  /Shading <<
    /SH01 <<
      /ShadingType 2
      /ColorSpace @Orange
      /Coords     [0 0 320 20]
      /Extend     [true true]
      /Function   << /FunctionType 2 /Domain [0 1] /N 1.0 >>
    >>
  >>
>>}
\special{pdf:content 0 0 320 20 re W n /SH01 sh}
\end{lstlisting}
where the ``Orange'' separation color space defined before is used again.
This example shows an axial shading (\code{ShadingType} 2) pattern.
\begin{center}
\makebox[320pt][l]{
  \special{pdf:put @resources <<
    /Shading <<
      /SH01 <<
        /ShadingType 2
        /ColorSpace @Orange
        /Coords     [0 0 320 20]
        /Extend     [true true]
        /Function   << /FunctionType 2 /Domain [0 1] /N 1.0 >>
      >>
    >>
  >>}
  \raisebox{-20pt}[0pt][20pt]{\special{pdf:content 0 0 320 20 re W n /SH01 sh}}
}
\end{center}

The shading pattern requires coordinate values to be mapped into color values.
Type 2 (Exponential Interpolation) \keyword{Function} is used for defining how
this mapping should occur here.
The above example, with the exponent $N=1$, is just a simple linear-gradient.

The final examples is a \keyword{tiling pattern}.
\begin{lstlisting}
\special{pdf:stream @MyPattern
    (0.16 0 0 0.16 0 0 cm 4 w
     50 0 m 50 28 28 50 0 50 c S 100 50
     m 72 50 50 28 50 0 c S
     50 100 m 50 72 72 50 100 50 c S
     0 50 m 28 50 50 72 50 100 c S
     100 50 m 100 78 78 100 50 100 c 22 100 0 78 0 50 c
     0 22 22 0 50 0 c 78 0 100 22 100 50 c S
     0 0 m 20 10 25 5 25 0 c f 0 0 m 10 20 5 25 0 25 c f
     100 0 m 80 10 75 5 75 0 c f
     100 0 m 90 20 95 25 100 25 c f
     100 100 m 80 90 75 95 75 100 c f
     100 100 m 90 80 95 75 100 75 c f
     0 100 m 20 90 25 95 25 100 c f
     0 100 m 10 80 5 75 0 75 c f
     50 50 m 70 60 75 55 75 50 c 75 45 70 40 50 50 c f
     50 50 m 60 70 55 75 50 75 c 45 75 40 70 50 50 c f
     50 50 m 30 60 25 55 25 50 c
     25 45 30 40 50 50 c f
     50 50 m 60 30 55 25 50 25 c 45 25 40 30 50 50 c f)
    <<
        /BBox [0 0 16 16]
        /PaintType 2
        /PatternType 1
        /Resources <<
            /ProcSet [/PDF]
        >>
        /TilingType 3
        /Type /Pattern
        /XStep 16
        /YStep 16
    >>
}
\end{lstlisting}

The above example defines a tiling pattern. The content stream containing
painting operators, \code{m} for ``move-to'', \code{c} for ``curve-to'',
\code{f} for ``fill'', and \code{S} for ``stroke'', defines the appearance of
the \keyword{pattern cell} for this tiling pattern. With the following code,
\begin{lstlisting}
\special{pdf:put @resources
    <<
        /ColorSpace << /CS01 [/Pattern /DeviceRGB] >>
        /Pattern << /PT01 @MyPattern >>
    >>
}
\special{pdf:content
    q 0.8 0.3 0.3 RG /CS01 cs 0.8 0.3 0.3 /PT01 scn
    0 0 320 100 re f
}
\end{lstlisting}
a box filled with the tiling pattern defined above is drawn.
\special{pdf:stream @MyPattern
    (0.16 0 0 0.16 0 0 cm 4 w
     50 0 m 50 28 28 50 0 50 c S 100 50
     m 72 50 50 28 50 0 c S
     50 100 m 50 72 72 50 100 50 c S
     0 50 m 28 50 50 72 50 100 c S
     100 50 m 100 78 78 100 50 100 c 22 100 0 78 0 50 c
     0 22 22 0 50 0 c 78 0 100 22 100 50 c S
     0 0 m 20 10 25 5 25 0 c f 0 0 m 10 20 5 25 0 25 c f
     100 0 m 80 10 75 5 75 0 c f
     100 0 m 90 20 95 25 100 25 c f
     100 100 m 80 90 75 95 75 100 c f
     100 100 m 90 80 95 75 100 75 c f
     0 100 m 20 90 25 95 25 100 c f
     0 100 m 10 80 5 75 0 75 c f
     50 50 m 70 60 75 55 75 50 c 75 45 70 40 50 50 c f
     50 50 m 60 70 55 75 50 75 c 45 75 40 70 50 50 c f
     50 50 m 30 60 25 55 25 50 c
     25 45 30 40 50 50 c f
     50 50 m 60 30 55 25 50 25 c 45 25 40 30 50 50 c f)
    <<
        /BBox [0 0 16 16]
        /PaintType 2
        /PatternType 1
        /Resources <<
            /ProcSet [/PDF]
        >>
        /TilingType 3
        /Type /Pattern
        /XStep 16
        /YStep 16
    >>
}%
\begin{center}
\raisebox{-100bp}[0bp][100bp]{\makebox[320bp][l]{\special{pdf:put @resources
    <<
        /ColorSpace << /CS01 [/Pattern /DeviceRGB] >>
        /Pattern << /PT01 @MyPattern >>
    >>
}%
\special{pdf:content
    0.8 0.3 0.3 RG /CS01 cs 0.8 0.3 0.3 /PT01 scn
    0 0 320 100 re f
}}}%
\end{center}

\subsubsection{Transparency}

%%EXAMPLE: Transparency
\XeTeX's transparency feature is currently lost in \xdvipdfmx, but the same
effect can be achieved by setting graphics state parameters with
\code{ExtGState} resources and \code{gs} operator. Here is a simple
transparency example:
\special{pdf:obj @gs01 <<
  /Type /ExtGState /CA 0.5 /ca 0.5 /AIS false
>>}%
\begin{figure}[ht]
\centering
\mbox{%
  %\raisebox{120pt}{\parbox[t]{0.8\textwidth}{%
  %   \addfontfeature{Color=444444}\lipsum[1]}%
  %}%
  %\hspace{-0.8\textwidth}%
  \special{pdf:put @resources <<
    /ExtGState << /GS01 @gs01 >>
  >>}%
  \fontsize{220pt}{220pt}\selectfont
  \special{pdf:code q /GS01 gs 1.0 0.8 0.2 rg}%
  α%
  \special{pdf:code 0.4 0.8 0.4 rg}%
  \hspace{-0.3em}%
  β%
  \hspace{-0.3em}\raisebox{0.5ex}{%
    \special{pdf:code 0.4 0.4 0.8 rg}%
    π%
  }%
  \special{pdf:code 0.6 0.2 0.8 rg}%
  \hspace{-0.2em}%
  γ%
  \special{pdf:code Q}%
}
\end{figure}
\begin{lstlisting}
\special{pdf:obj @GS01 <<
  /Type /ExtGState /CA 0.5 /ca 0.5 /AIS false
>>}%
\mbox{%
  \special{pdf:put @resources <<
    /ExtGState << /GS01 @GS01 >>
  >>}%
  \special{pdf:code q /GS01 gs 1.0 0.8 0.2 rg}%
  α%
  \special{pdf:code 0.4 0.8 0.4 rg}%
  \hspace{-0.3em}%
  β%
  \hspace{-0.3em}\raisebox{0.5ex}{%
    \special{pdf:code 0.4 0.4 0.8 rg}%
    π%
  }%
  \special{pdf:code 1.0 0.2 0.4 rg}%
  \hspace{-0.2em}%
  γ%
  \special{pdf:code Q}%
}
\end{lstlisting}
where values for \code{CA} and \code{ca} represent opacity of stroke and
fill color respectively. Again, \code{pdf:put} command must go into the same
page as subsequent graphics and text drawing operators.

\section{Dvipdfmx Extensions}

\newfeature{Addition in \TeX\ Live 2016}

\noindent{}A new special \code{dvipdfmx:config} was introduced in
\TeX Live 2016 which makes it possible to invoke a command line option.
Several single letter command line options except \option{D} are supported.
For example,
\begin{lstlisting}
dvipdfmx:config C 0x10
\end{lstlisting}
sets the \dvipdfmx's compatibility flags. See, the section
``\hyperref[SEC:compatibility]{Incompatible Changes}'' for an explanation of
compatibility flags.

\section{PS Specials}

PS (PostScript) specials can be used to insert a raw PostScript code for drawing
graphics objects and transforming subsequent text and graphics.
Please note that support for PostScript operators in \dvipdfmx\ is very
limited. It is just enough for interpreting PostScript figures output by \MP.
Only a basic set of operators for arithmetic and math, stack operation and manipulation,
graphics state, path construction and painting, glyph and font, are supported.
See, Table~\ref{TABLE:PS} for the list of
recognized PostScript operators.

\begin{table}
    \centering
    \begin{tabular}{p{3cm}>{\raggedright\arraybackslash}p{8cm}}\hline
        Classification & Operators \\ \hline\hline
        Arithmetic \& Math & \code{add} \code{sub} \code{mul} \code{div} \code{neg} \code{truncate}\\
        Stack Operation & \code{clear} \code{pop} \code{exch}\\
        Graphis State & \code{gsave} \code{grestore}
          \code{setlinewidth} \code{setdash} \code{setlinecap} \code{setlinejoin} \code{setmiterlimit} \code{setgray} \code{setrgbcolor} \code{setcmykcolor} \\
        Coordinate System & \code{concat} \code{scale} \code{translate} \code{rotate} \code{idtransform} \code{dtransform}\\
        Path Construction & \code{currentpoint} \code{newpath} \code{closepath} \code{moveto} \code{rmoveto} \code{lineto} \code{rlineto} \code{curveto} \code{rcurveto} \code{arc} \code{arcn} \code{clip} \code{eoclip} \\
        Painting & \code{stroke} \code{fill} \\
        Glyph \& Font & \code{show} \code{findfont} \code{scalefont} \code{setfont} \code{currentfont} \code{stringwidth}\\
        \hline
    \end{tabular}
    \caption{List of PostScript operators recognized by \dvipdfmx.}%
    \label{TABLE:PS}
\end{table}

It might be enough for the purpose of basic graphics drawings but as there are
no support for conditionals and controls it is not enough for complicated tasks,
especially, the PSTricks package is not supported.

In \dvipdfmx, text handling is extended to support CJK text.
The following code draws Japanese text like shown in Figure~\ref{FIG:verttext}:
\begin{lstlisting}
\special{pdf:mapline uprml UniJIS-UTF8-H yumindb.ttf}
\special{ps: uprml findfont 16 scalefont setfont
  currentpoint moveto
  (...some Japanese text goes here...) show
}
\end{lstlisting}

\chapter{Fonts and Encodings}

\section{Fonts and Encodings Support}

In \dvipdfmx, all font formats supported by \dvipdfm\ are also supported with
many improvements: The CFF conversion for PostScript Type1 fonts\footnote{PostScript
Type1 font support is restricted to the binary format as in \dvipdfm.} is
implemented which greatly reduces the output file size. Embedded TrueType fonts are
now subsetted. The OpenType font format is also supported.\footnote{Its
implementation is based on the OpenType specification version 1.4.
Newly added features such as color fonts and variable fonts are not supported yet.}

There are various enhancements made for supporting Unicode and legacy multi-byte
character encodings for East Asian languages.


\section{Font Mappings}

The Syntax of font-mapping (fontmap) files is basically the same as in \dvipdfm.
There are few extensions available in \dvipdfmx. In addition to the 8-bit \code{enc}
file and keywords \code{builtin} and \code{none}, \dvipdfmx\ accepts a PostScript CMap
Resource name and the keyword \code{unicode} in the encoding field.

When the keyword \code{unicode} is specified in the encoding field of fontmap
files, it is assumed that Unicode values are used in the input DVI file.

\begin{lstlisting}
  urml    unicode  SourceHanSerifJP-Light.otf
\end{lstlisting}

Although the DVI format allows 3-byte and 4-byte character codes to be used,
\dvipdfmx\ only supports up to 2-byte range since there is no TFM format supporting
3-byte or 4-byte codes.

For PostScript Type1 fonts which do not support Unicode natively, an auxiliary file,
the Adobe Glyph List, is required to make it possible to use fonts with Unicode access.

As a general framework for supporting legacy multi-byte encodings, \dvipdfmx\ employs
PostScript CMap Resources for handling input strings encoded in various
character encodings. A CMap name can be specified in the encoding field just like
the encoding name for 8-bit encodings. For example, to specify the CMap ``UniJIS-UCS2-H'',

\begin{lstlisting}
  urml    UniJIS-UCS2-H  HiraMinPro-W3.otf
\end{lstlisting}

For information on the Adobe Glyph List and PostScript CMap Resources, see,
the section \ref{SEC:auxfiles}, ``Auxiliary Files.''

\subsection{Extended Syntax and Options}

Few options are available in \dvipdfmx\ in addition to the original dvipdfm's
one. Please note that all features which makes \dvipdfmx\ to use
non-embedded fonts are deprecated, as by doing so it makes \dvipdfmx\ to create
PDF files which can be non-compliant to the ISO standards.

\subsubsection{SFD Specification}

For bundling up a font split into multiple subfonts via SFD back into
a single font, dvipdfmx supports extended syntax of the form
\begin{lstlisting}
tfm_name@SFD@  encoding  filename  options
\end{lstlisting}
A typical example looks like:
\begin{lstlisting}
gbsn@EUC@  GB-EUC-H  gbsn00lp
\end{lstlisting}
where TFMs \code{gbsn00}, \code{gbsn01}, \code{gbsn02}... are mapped into a
single font named \code{gbsn00lp} via the rule described in the SFD file
\code{EUC}.

\subsubsection{TrueType Collection Index}

TrueType Collection index number can be specified with \code{:n:}
in front of the TrueType font name:
\begin{lstlisting}
min10  H  :1:mincho
\end{lstlisting}
In this example, the option \code{:1:} tells \dvipdfmx\ to select first
TrueType font from the TTC font \code{mincho.ttc}. Alternatively, the
\option{-i} option can be used in the option field to specify TTC index:
\begin{lstlisting}
min10  H  mincho  -i 1
\end{lstlisting}

\subsubsection{Non-embedding Switch}

\deprecated{Use of this option is deprecated.}
\noindent{}The character \option{!} in front of the font name can be used to
indicate that the font shall not be embedded. This feature greatly reduces the
size of the final PDF output, but the PDF file may not be viewed exactly the same
in other systems on which appropriate fonts are not installed.
\bigskip

\noindent{}NOTE: \dvipdfmx\ always converts input encodings to CIDs and then
uses Identity CMaps\footnote{Predefined CMaps \code{Identity-H} and
\code{Identity-V} for the identity mapping.}
in the output PDF. However, \lnum{ISO~32000-1:2008} describes as
\begin{quoting}
The Identity-H and Identity-V CMaps shall not be used with a non-embedded font.
Only standardized character sets may be used.
\end{quoting}
which had never appeared in Adobe's PDF References. This makes all PDF files
generated by \dvipdfmx\ with non-embedded CID-keyed fonts non-compliant to
the ISO standards.

\subsubsection{`Standard' CJK Fonts}

\deprecated{This feature is deprecated.}
\noindent{}Use of this feature shall be avoided for new documents. It is
described here since it might still be useful for some situations.

\dvipdfmx\ recognizes several `Standard' CJK fonts although there are no such
notion in PDF. In older days where there were not so many freely available CJK
fonts, it was sometimes useful to create PDF files without embedding fonts and
let PDF viewers or printers to use substitute fonts (tend to be higher quality)
installed in their systems. \dvipdfmx\ `knows' several fonts which might be
available in PostScript printers and PDF applications such as Acrobat Reader,
and uses them without actually having it.
See, Table~\ref{TABLE:StdCJKFont}, for the list of available `Standard' CJK
fonts.

\begin{table}
    \centering
    \begin{tabular}{lll}\hline
    Character Collection & Font Family & Description \\ \hline\hline
    Adobe-Japan1 & Ryumin-Light & PS printers \\
        & GothicBBB-Medium & \\
    Adobe-CNS1 & MHei-Medium-Acro & Acrobat Reader 4 \\
        & MSung-Light-Acro & \\
    Adobe-GB1 & STSong-Light-Acro & \\
        & STHeiti-Regular-Acro & \\
    Adobe-Japan1 & HeiseiMin-W3-Acro & \\
        & HeiseiKakuGO-W5-Acro & \\
    Adobe-Korea1 & HYGoThic-Medium-Acro & \\
        & HYSMyeongJo-Medium-Acro & \\
    Adobe-CNS1 & MSungStd-Light-Acro & Acrobat Reader 5 \\
    Adobe-GB1 & STSongStd-Light-Acro & \\
    Adobe-Korea1 & HYSMyeongJoStd-Medium-Acro \\
    Adobe-CNS1 & AdobeMingStd-Light-Acro & Adobe Reader 6 \\
    Adobe-GB1 & AdobeSongStd-Light-Acro & \\
    Adobe-Japan1 & KozMinPro-Regular-Acro & \\
        & KozGoPro-Medium-Acro & \\
    Adobe-Korea1 & AdobeMyungjoStd-Medium-Acro & \\
    Adobe-CNS1 & AdobeHeitiStd-Regular & Adobe Reader 7 \\
    Adobe-Japan1 & KozMinProVI-Regular & Adobe Reader 8\\
    \hline
    \end{tabular}
    \caption{List of available `Standard' CJK font. Most of them are
    available as a part of Adobe Asian Font Packs for each versions of
    Adobe or Acrobat Reader.}\label{TABLE:StdCJKFont}
\end{table}

Only fixed-pitch glyphs (i.e., quarter, third, half, and full widths) are
supported for those fonts.

\subsubsection{Stylistic Variants}

\deprecated{Use of this option is deprecated.}
\noindent{}Keywords \code{,Bold}, \code{,Italic}, and \code{,BoldItalic} can be
used to create synthetic bold, italic, and bolditalic style variants from other
font using PDF viewer's (or OS's) function.
\begin{lstlisting}
jbtmo@UKS@  UniKSCms-UCS2-H  :0:!batang,Italic
\end{lstlisting}

Availability of this feature highly depends on the implementation of PDF
viewers. This feature is usually not supported for embedded fonts.
Notice that this option automatically disables font embedding thus
use of it is deprecated.

\subsection{Specifying Unicode Plane}

As there are no existing TFM formats supporting 3-byte or 4-byte character
encodings, the only way to use Unicode characters other than the BMP is to
map the code range 0-65535 to different planes via (e.g., to plane 1)
the \option{-p 1} fontmap option. This option is
available only when \code{unicode} is specified in the encoding field.

\subsection{OpenType Layout Feature}

The OpenType Layout Feature fontmap options mentioned below are only meaningful
when \code{unicode} is specified in the encoding field.

With the \option{-w} option, writing mode can be specified.
\option{-w 1} denotes the font is for vertical writing. It automatically
enables an OpenType Layout Feature related to vertical writing, namely,
\code{vert} or \code{vrt2}, to choose proper glyphs for vertical text.

\newfeature{Addition in \TeX\ Live 2017.}

The \option{-l} (lower case el) option can be used to enable various
OpenType Layout GSUB Features. For example, \option{-l jp04} enables
\code{jp04} feature to select \lnum{JIS2004} forms for Kanjis.
Features can be specified as a ``:'' separated list of OpenType Layout
Feature tags like \option{-l vkna:jp04}. Script and language may be
additionally specified as
\option{-l kana.JAN.ruby}.

An example can be
\begin{lstlisting}
uprml-v  unicode  SourceHanSerifJP-Light.otf  -w 1 -l jp90
\end{lstlisting}
which declares that font should be treated as for vertical writing and
use \lnum{JIS1990} form for Kanjis.

\begin{figure}
\centering
\jpzerofourexamples\hspace{30pt}\jpninezeroexamples%
\caption{\lnum{JIS2004} vs. \lnum{JIS1990} form.}\label{FIG:jp90}
\end{figure}

This feature is limited to the single substitution, there are no way to select
a glyph from multiple candidates, such as in the \code{aalt} feature, and specifying
general many-to-many glyph substitutions does not take effect.

\section{Other Improvements}

This section briefly describes other improvements made for \dvipdfmx.
There is an extension to glyph name handling in the \code{enc} file for
seamless support of both PostScript Type1 and TrueType fonts.
PostScript Type1 font support is enhanced although this format might be
considered obsolete.

\subsection{Extended Glyph Name Syntax}

\dvipdfmx\ accepts the following syntax for glyph names in the \code{enc} file:
\code{uni0130}, \code{zero.onum} and \code{T\_h.liga}.
Each represents a glyph accessed with Unicode value \code{U+0130},
oldstyle number for zero and ``Th'' ligature accessed via the OpenType
Layout GSUB Feature \code{onum} and \code{liga}, respectively.
Note that \dvipdfmx\ does not understand glyph names which directly
use a glyph index such as \code{index0102} or \code{gid2104}.

When \dvipdfmx\ encounters a glyph name, e.g., \code{T\_h.liga}, it first looks
for OpenType \code{post} table if such glyph name exists; if it exists, then
\dvipdfmx\ simply uses \code{post} table and maps the glyph name to the glyph index;
if not, \dvipdfmx\ tries to convert \code{T\_h} to a Unicode sequence (U+0054
U+0068 in this example) via the AGL mapping; then, OpenType \code{cmap} table is used
to further convert the resulting Unicode sequence to the sequence of glyph indices;
finally, the OpenType Layout Feature \code{liga} is applied to get the desired glyph.

A glyph name of a form \code{a.swsh2} can be specified to denote the 2nd swash
variant form of the letter `a'.

\subsection{CFF Conversion}

\dvipdfmx\ supports on-the-fly PostScript Type1 to CFF (Type1C) conversion
which greatly reduces size of the resulting PDF file when using Type1 fonts.
Conversion is essentially `lossless' and there should not be any quality loss.
However, due to differences in the ability of rasterizers, there might be
noticeable differences on rendering result.

When an older Type1 font is used, \dvipdfmx\ may give the following warning
message:
\begin{lstlisting}
Obsolete four arguments of "endchar" will be used for Type1
"seac" operator.
\end{lstlisting}
It happens when there is an accented character made as a composite glyph using
the ``seac'' operator.
This warning is issued as conversion can't be done without relying on the
\emph{deprecated} usage of the \code{endchar} operator. However, as mentioned in
``Appendix C Compatibility and Deprecated Operators'' of Adobe Technical
Note \#5177,
``\href{http://wwwimages.adobe.com/content/dam/Adobe/en/devnet/font/pdfs/5177.Type2.pdf}{Type 2 Charstring Format}'',
PDF applications should support this operator and hence this warning
message can be ignored.

Use of Type1 font should be avoided as much as possible.
Please consider using OpenType version instead.

\section{Font Licensing}

In OpenType font format, information regarding how a font should be treated
when creating a document can be recorded.%
\footnote{See,
``\href{http://www.microsoft.com/typography/otspec/os2.htm}{OpenType Specification:
OS/2 -- OS/2 and Windows Metrics Table}''.}
\dvipdfmx\ uses this information to decide whether font embedding is permitted.

This font licensing information is indicated by the flag called \code{fsType}
recorded in OpenType font files; each bits representing different restrictions on
font embedding. If multiple flag bits are set in \code{fsType}, the least restrictive
license granted takes precedence in \dvipdfmx.
The \code{fsType} flag bit recognized by \dvipdfmx\ is as follows:
\begin{itemize}
    \item Installable embedding
    \item Editable embedding
    \item Embedding for Preview \& Print only
\end{itemize}
\dvipdfmx\ issues the following warning message for fonts with `Preview \& Print only'
setting:
\begin{verbatim}
  This document contains 'Preview & Print' only licensed font
\end{verbatim}

For a font with this type of licensing, font embedding is allowed solely for the
purpose of (on-screen) viewing and/or printing; further editing of the document
or extracting embedded font data for other purposes are not allowed.
One way to ensure this condition is to protect your document with a non-empty
password.

All other flags are treated as more restrictive license than any of
the above flags and treated as ``No embedding allowed''; e.g., if both
of the editable-embedding flag and unrecognized license flag is set,
the font is treated as editable-embedding allowed, however, if only
unrecognized flags are set, the font is not embedded.

Font Embedding flags are preserved in the embedded font if they are embedded
as a TrueType font or a CIDFontType2 CID-keyed font.
For all fonts embedded as a PostScript font (Type1C and CIDFontType0
CID-keyed font), they are not preserved.
Only \code{Copyright} and \code{Notice} in the \keyword{FontInfo} dictionary
are preserved in this case.

Some font vendors put different embedding restrictions for different
condition; e.g., font embedding might not be permitted for the commercial
use unless you acquire the ``commercial license'' separately.
Please read EULA carefully before making decision on the font usage.

See, for example,
\href{http://www.adobe.com/products/type/font-licensing/font-embedding-permissions.html}{Adobe's site on font embedding permissions}
for the font in the Adobe Type Library.
Microsoft also has a
\href{http://www.microsoft.com/typography/RedistributionFAQ.mspx}{FAQ page on Font Redistribution}.

For Japanese font in general, embedding permission tend to be somewhat
restrictive. Japanese users should read the statement regarding font
embedding from Japan Typography Association (in Japanese):\medskip

\url{http://www.typography.or.jp/act/morals/moral4.html}
\medskip

\dvipdfmx\ does not support full embedding. Only subset embedding is supported.

\chapter{Encryption}

\section{Encryption Support}\label{SEC:encryption}

\dvipdfmx\ offers basic PDF password security support including the 256-bit AES encryption.
Only the ``Standard'' security handler is supported and the Public-key security handler is not
supported.

Encryption is enabled by \option{-S} command line option.
\begin{lstlisting}
  dvipdfmx -S -K 128 -P 0x14
\end{lstlisting}
where \code{-K} and \code{-P} options are used to specify encryption key length and
permission flags respectively, and are briefly explained in Table~\ref{TABLE:enc-options}.

\begin{table}
  \centering
  \begin{tabular}{lp{8cm}}\hline
      Option & Description \\ \hline\hline
      \code{-S} & Enable PDF encryption. \\
      \code{-K} \textit{number} & Set encryption key length. The default value
      is 40.\\
      \code{-P} \textit{number} & Set permission flags for PDF encryption.
      The \textit{number} is a 32-bit unsigned integer representing permission
      flags.
      See, Table~\ref{TABLE:flags}. The default value is \code{0x003C}.\\
      \hline
  \end{tabular}
  \caption{Command line options for encryption.}%
  \label{TABLE:enc-options}
\end{table}

When \dvipdfmx\ is invoked with encryption via the \code{-S} option,
passwords will be asked.
However, in some circumstances, it might be desirable to avoid being prompted for
passwords. In that case, use the \code{pdf:encrypt} special to supply passwords in
the \TeX\ file, as,
\begin{lstlisting}
  \special{pdf:encrypt userpw (foo) ownerpw (bar) length 128 perm 20}
\end{lstlisting}
Here, user and owner passwords are supplied as PDF string objects (\code{foo} and \code{bar}
in the example above) which can be empty.

Up to two passwords can be specified for a document -- an owner password and a user password.
If a user attempts to open an encrypted document with user password being set, PDF application
should prompt for a password. Users are allowed to access the contents of the document only when
either password is correctly supplied.
Depending on which password (user or owner) was supplied, additional operations
allowed for the opened document is determined; full access for users who opened
with the correct owner password or additional operations
controlled by permission flags for users who opened with the correct user password.

Although PDF specification allows various character encodings other than \code{US-ASCII}
for entering password, \dvipdfmx\ is unable to handle it properly.
Thus it must be assumed that \code{US-ASCII} is used for password strings.

Access permission flags can be specified via \option{-P} command-line option.
Each bits of (32-bit unsigned) integer number given to this option represents user
access permissions; e.g., bit position 3 for allowing ``print'', 4 for
``modify'', 5 for ``copy or extract'', and so on. See, Table~\ref{TABLE:flags}.
For example, \code{-P 0x34} allows printing, copying and extraction of text, and adding and
modifying text annotation and filling in interactive form fields (but disallows
modification of the contents of the document).
\begin{table}
    \centering
    \begin{tabular}{lp{8.3cm}}\hline
        Bit Position & Meaning \\ \hline\hline
        3 & (Revision 2) Print the document. \\
          & (Revision 3 or greater) Print the document. Print quality depending on bit 12.\\
        4 & Modify the contents of the document by operations other than those controlled
            by bits 6, 9, and 11. \\
        5 & Copy or extract text and graphics from the document. \\
        6 & Add or modify text annotations, fill in interactive form fields.
            Creation and modification of interactive form field is also
            allowed if bit 4 is set.\\
        9 & (Revision 3 or greater) Fill in existing interactive form fields
            (including signature fields), even if bit 6 is clear.\\
        10  & \em{Deprecated in PDF 2.0} \\
            & (Revision 3 or greater) Extract text and graphics (in support of accessibility to
              users with disabilities or for other purposes).\\
        11  & (Revision 3 or greater) Assemble the document
              (insert, rotate, or delete pages and create document outline items or thumbnail
              images), even if bit 4 is clear.\\
        12  & (Revision 3 or greater) High-quality printing.
              When this bit is clear (and bit 3 is set), printing shall be limited to a low-level,
              possibly of degraded quality.\\
        \hline
    \end{tabular}
    \caption{Flag bits and their short explanation.
            Revision 2 is used when encryption key length is 40 bits or when PDF output
            version is less than 1.5. Otherwise, Revision 3 or greater is used.}\label{TABLE:flags}
\end{table}

The \option{-K} option can be used to specify the encryption key length.
The key length must be multiple of 8 in the range 40 to 128, or 256 (for PDF
version 1.7 plus Adobe Extension or PDF version 2.0). Please note
that when key length 256 is specified for PDF version 1.7 output, it requires
Adobe's Extension to \lnum{PDF-1.7} and hence PDF applications may not support it.
PDF version 1.4 is required for key length more than 40 bits. AES encryption algorithm requires
PDF version 1.6.

To show some examples:\\
128-bit AES encryption with print-only (high-quality) setting,
\begin{lstlisting}
  dvipdfmx -V 5 -S -K 128 -P 0x804 input.dvi
\end{lstlisting}
256-bit AES encryption with print (low-quality), adding and modifying text annotations
allowed,
\begin{lstlisting}
  dvipdfmx -V 2.0 -S -K 256 -P 0x24 input.dvi
\end{lstlisting}

The default value for \option{-K} is 40 and for \option{-P} is \code{0x003C0}
(all bits from bit-position 3 to 6 set).

\chapter{Compatibility}

\section{Incompatible Changes}\label{SEC:compatibility}

There are various minor incompatible changes to \dvipdfm.

The \option{-C} command line option may be used for compatibility to
\dvipdfm\ or older versions of \dvipdfmx. The \option{-C} option takes flags
meaning
\begin{itemize}
    \item bit position 2: Use semi-transparent filling for tpic shading
    command, instead of opaque gray color. (requires PDF 1.4)
    \item bit position 3: Treat all CID-keyed font as fixed-pitch font. This is
    only for compatibility.
    \item bit position 4: Do not replace duplicate fontmap entries.
    \dvipdfm\ behavior.
    \item bit position 5: Do not optimize PDF destinations. Use this if you
    want to refer from other files to destinations in the current file.
    \item bit position 6: Do not use predictor filter for Flate compression.
    \item bit position 7: Do not use object stream.
\end{itemize}

The remap option \option{-r} in fontmaps is no longer supported and is
silently ignored. The command line option \option{-e} to disable partial
(subset) font embedding is not supported.

\section{Important Changes}

Here is a list of important changes since the \TeX\ Live 2016 release:
\begin{itemize}
\item Changes to make PDF/A creation easier: Always write CIDSet and CharSet
for embedded fonts. Do not compress XMP metadata.
\item Merge from libdpx for p\TeX-ng by Clerk Ma.
\item Addition of \code{STHeiti-Regular-Acro} for CJK `Standard' fonts.
\item Command line option \option{-p} takes precedence over \code{papersize}
and \code{pagesize} specials.
\item Fixed serious bugs in supporting `\code{unicode}' encoding:
OpenType Layout Feature \code{vert} and \code{vrt2} was not enabled.
Support for format 2 CFF charsets was broken.
\item Added simplified version of OpenType Layout support: The `\option{-l}'
option in fontmaps.
\end{itemize}
The full \code{ChangeLog} entries can be viewed via the web interface of the
\TeX\ Live SVN repository:
\medskip

\url{http://www.tug.org/svn/texlive/trunk/Build/source/texk/dvipdfm-x}
\medskip

There was an undocumented feature for supporting OpenType Layout but it was
dropped. Simplified support for the OpenType Layout was introduced instead.


%\renewcommand{\refname}{Further Reading}
\renewcommand{\bibname}{Further Reading}
\begin{thebibliography}{99}
\bibitem{DVIPDFM}  ''\href{http://mirrors.ctan.org/dviware/dvipdfm/dvipdfm.pdf}%
{Dvipdfm User's Manual}'' written by Mark~A.~Wicks.
\bibitem{ADOBE} Adobe's PDF References and a free copy of
\lnum{ISO 32000-1:2008} standard are available from
``\href{http://www.adobe.com/devnet/pdf.html}{PDF Technology Center}''
on \href{http://www.adobe.com/devnet.html}{Adobe Developer Connection}.
\bibitem{MICROSOFT} The OpenType Specification is available from Microsoft's
site:
``\href{http://www.microsoft.com/en-us/Typography/OpenTypeSpecification.aspx}%
{OpenType Specification}''.
\bibitem{PNGSPEC} ''\href{https://www.w3.org/TR/2003/REC-PNG-20031110/}%
{Portable Network Graphics (PNG) Specification (Second Edition)}''.
\bibitem{CHOF} An article regarding DVI specials: Jin-Hwan Cho,
``\href{http://www.tug.org/TUGboat/tb30-1/tb94cho.pdf}{DVI specials for PDF generation}'',
TUGboat, 30(1):6-11, 2009.
\end{thebibliography}

\appendix

\chapter{GNU Free Documentation License}\label{SEC:FDL}

This document is distributed under the term of the GNU Free Documentation
License. See, the attached file for copying conditions.%
\marginnote{%
\special{pdf:fstream @fileobj (fdl-1.3.txt)}%
\special{pdf:ann width 10bp height 20bp
   << /Type /Annot
      /Subtype /FileAttachment
      /FS <<
            /Type /Filespec
            /F    (fdl-1.3.txt)
            /EF   << /F @fileobj >>
          >>
      /Name /PushPin
      /C [0.8 0.2 0.2]
      /T (GNU Free Documentation License)
      /Subj (GNU FDL)
      /Contents (Plain text version of the GNU Free Documentation License.)
   >>
}}%

Or, in case that PDF viewers can not extract attached files, please visit the
following site:
\medskip

\url{http://www.gnu.org/licenses/fdl.html}


\end{document}
